\documentclass[a4paper,12pt]{refrep}
\title{aa-kernel}
\usepackage{color}
\definecolor{light-gray}{gray}{0.95}
\newcommand{\code}[1]{\colorbox{light-gray}{\texttt{#1}}}

\begin{document}
\chapter{\texttt{aa-kernel}}
\code{aa-kernel} is an embedded operating system for 32 and 64-bit 
ARM CPUs. The kernel consists of:
\begin{itemize}
\item A base set of subsystems and supporting code, 
    located in \code{src/boot} and \code{src/kernel}.
\item Architecture and platform-specific subsystems, 
    located in \code{src/arch} and \code{src/platform}.
\item Drivers not limited to any one architecture or platform, 
    such as the SDHCI driver, located in \code{src/drivers}.
\end{itemize}

\subsection{Design Goals}
Much of the work on \code{aa-kernel} was carried out in order 
to support FPGA development 
on Zynq systems. As a result, the Cortex A9 port is 
currently the most well-supported variant in terms of 
driver support and testing. 

Physical testing for this port was carried out on the 
Digilent Zybo board. The kernel is also capable of booting 
on the aarch64 virt platform, the Integrator CP, and the 
Raspberry Pi, though the Zynq and virt ports are the only 
currently tested ports.

\subsection{Boot Sequence}
On boot, on-board firmware or an off-board JTAG command sequence 
loads the kernel image into memory and, on a single core, 
jumps to the \code{\_start} symbol, defined in 
\code{src/arch/\{arch\}/start.s}. This \code{\_start} symbol 
is an assembly function that sets up the stack pointer, 
installs the base interrupt vectors, and calls the \code{init()} 
function located in \code{src/boot/init.c}.
\\
\code{init()} executes architecture-specific 
initialization functions (defined in the relevant \code{arch} 
folders) 
and initializes the buddy-block page allocator, slab allocator, 
random allocator, thread subsystem and IRQ subsystem. It then runs
\code{platform\_init()}, also defined in a specific \code{platform} 
folder, enables IRQs, and finally runs \code{kmain()}.

\newpage
\subsection{Subsystems}
{\bf IRQs {\bf \code{(src/kernel/irq.c)}}} \\
IRQs refer to any interrupt generated externally, such as from hardware.
The base IRQ handler structure is typically installed at the start of physical memory for ARM CPUs.
\\
\\
{\bf Timers } \\
Interrupt-driven timers enable scheduling events in time, allowing features such as preemptive multitasking.
A preset timer IRQ is defined per-platform in \\ \code{src/include/platform/\{platform\}/irq\_num.h} \\ and installed during \code{platform\_init()} at boot.
\\
\\
{\bf Buddy-Block Allocator {\bf \code{(src/kernel/bb\_alloc.c)}}} \\
The buddy-block allocator is the base of all other allocators and allows allocating $2^n$ pages of physically contiguous, page-aligned blocks of memory. 
\\
\\
{\bf Slab Allocator {\bf \code{(src/kernel/bb\_alloc.c)}}} \\
In many portions of the kernel, data structures of a 
static fixed size are repeatedly allocated during runtime. 
The slab allocator dedicates a block of memory to contiguous 
arrays of these structures for rapid allocation.
\\
\\
{\bf Random Allocator {\bf \code{(src/kernel/malloc.c)}}} \\
The random allocator defines \code{void *malloc(uint32\_t size)}, which allocates \code{size} bytes successfully or returns 0. 
\\
\\
{\bf Block Devices{\bf \code{(src/kernel/block\_cache.c)}}}
\\
\\
\\
{\bf Console {\bf \code{(src/kernel/console.c)}}}
\\
\\
\\
{\bf Multitasking {\bf \code{(src/kernel/thread.c)}}}




\subsection{Drivers}
{\bf PL011 Serial Console {\bf \code{(src/drivers/serial/pl011.c)}}}
{\bf PL390 Interrupt Controller {\bf \code{(src/drivers/serial/pl011.c)}}}
{\bf SD Host Controller {\bf \code{(src/drivers/block/sdhci.c)}}}


\end{document}
